\expandafter\ifx\csname ifdraft\endcsname\relax
 \documentclass{jsarticle}
 \begin{document}
\fi
%%%%%%%%%%%%%%%%%%%%%%%%%%%%%%%%%%%%%%%%%%%

\begin{center} {\bf 要 旨} \end{center}

LaTeXは、論文を執筆する上で、非常に強力な組版ツールである。テキストベースで記述される構造記述ファイル、美しく描画される数式、強力な文献管理システムなど......そしておまけにGit管理ができる。なんと便利な!挙げだしたらキリがない。情報工学の学位を修める者として、必ず習得しておきたいスタックの1つである。
本文書は、岡山理科大学 工学部 情報工学科 片山研究室における卒業論文向けの \LaTeX テンプレートである。本文書をベースとして君たちは卒業論文を書くことになる。
\TeX は、君たちが今まで触ってきた Microsoft Word や Googleドキュメント のそれとは全く異なり、WYSIWYGでない状態で編集しなけらばならない。しかし、ここで怯まないでほしい。君たちは訳のわからないまま "printf関数" を習得し、変数の表示を行い、エスケープシーケンスを使いこなして来た。そうでしょ?
\TeX も本質的には何も変わらない。一定のルールに則って自分の作りたい文章を組み立てて行くだけである。これを苦労なくできるようになった暁には、立派なWordアンチの誕生である。
さあ、この駄文を削除して、君の論文を書き始めるのだ。ほら早くキーボードに手を置いてShift + ↑で全選択して消すんだ。幸運を祈る。


\begin{center} {\bf キーワード} \end{center} \vspace{0.01em}
\TeX,\LaTeX,テキストベース組版システム,
卒業論文

%%%%%%%%%%%%%%%%%%%%%%%%%%%%%%%%%%%%%%%%%%%
\expandafter\ifx\csname ifdraft\endcsname\relax
  \end{document}
\fi