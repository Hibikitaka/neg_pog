\expandafter\ifx\csname ifdraft\endcsname\relax
 \documentclass{jsarticle}
 \begin{document}
\fi
%%%%%%%%%%%%%%%%%%%%%%%%%%%%%%%%%%%%%%%%%%%

\def\Path{./51_Conclusion}

%%%%%%%%%%%%%%%%%%%
\chapter{結 論}
%%%%%%%%%%%%%%%%%%%
\section{結論}
スケジューリング問題は私たちの生活の中に数多く存在する.特に製造業において効率的なスケジュールを立てることによって,製造効率の向上,コストの削減,納期の見積もりの目処が立てやすくなる.また長時間労働やあるいは人件費削減などが可能になる.このように効率化を行うことによって多くの問題の軽減や解決が可能になる.フローショップスケジューリング問題(Flow-shop Scheduling Problem, FSP)が挙げられる.
FSPは代表的な組み合わせ最適化問題の一つであり,NP-困難であるため,現実的な時間で良好な解を算出する近似解法の研究が盛んに行われている.

現在,この問題に対して遺伝的アルゴリズム(Genetic Algorithm, GA),反復貪欲法(Iterated Greedy, IG),焼きなまし法(Simulated Algorithm, SA)など,多くのメタ戦略アルゴリズムが適用されている.
近年,これらのアルゴリズムと同様に,渡り鳥の行動を基にしたアルゴリズムである,渡り鳥最適化アルゴリズム(Migrating Birds Optimization, MBO)が提案されており,二次割り当て問題(Quadratic Assignment Problem, QAP)に対する有効なメタ戦略アルゴリズムであることが知られている.

本論文では第\ref{chap:MBO_FSP}章で,FSPに対する解法として渡り鳥最適化アルゴリズム(Migrating Birds Optimization, MBO)を提案した.第\ref{chap:results}章では,性能比較実験としてベンチマーク問題例を使用して提案MBOの性能の評価するため実験を行った.比較解法はFSPに対して最も効率的な手法の一つである反復貪欲法(Iterated Greedy, IG)とした.結果から従来IGは提案MBOに比べて平均的に良好な解を示した.
 
%%%%%%%%%%%%%%%%%%%%%%%%%%%%%%%%%%%%%%%%%%% 
 
\section{今後の課題}
今後の課題としては,以下の2点が考えられている.

一つ目は局所探索法の適用が考えられる.従来MBOは大域的探索に優れているが局所的探索の性能は十分でない.そこでMBOに局所探索性能の向上のため局所探索を導入することでMBOの解の精度がよくなると推測する.しかし,MBOでは良好な解の精度を得ようとすると解の集団が大きくなる傾向になる.これらすべての解に局所探索を行うと局所探索の時間を多く要するため,近傍生成の回数が極端に減り解の精度が著しく落ちてしまう.そこで一定の確率で局所探索を行ったリ,一部に対して局所探索を行うなど,局所探索の適用方法や解の数を調整する必要がある.MBOに対する局所探索の適用については付録Dに示す.またFSPに対して極めて効率的な解法にIGがあり,IGの中でも良好な解を示しているFernandez-Viagas と Framinanらの提案したIG~\cite{fernandez2019best}ではジョブを取り除いた時にも局所探索を適用している.これは実行不可能解に対しても局所探索を行うことで解に多様性を与え,解の精度を上げていると考える.
% 一つ目は局所探索の適用が考えられる.現在のMBOには局所探索が適用されていない.従来法であるIGでは局所探索を適用することで解の精度が大幅に向上した.従来IGの近傍操作をMBOに適用することで解の精度が向上したことから局所探索も適用することによって,MBOの解の精度がよくなると推測する.しかし,MBOでは良好な解の精度を得ようとすると解の集団が大きくなる傾向になる.これらすべての解に局所探索を行うと局所探索の時間を多く要するため,近傍生成の回数が極端に減り解の精度が著しく落ちてしまう.そこで一定の確率で局所探索を行ったリ,一部に対して局所探索を行うなど,局所探索の適用方法も調整する必要がある.

二つ目は多様性の維持が挙げられる.
前述した通り,MBOで解の精度を上げようとすると,ある程度解の集団を大きくする必要がある.
この理由として,解の集団を大きくすることで集団の多様性を保ち,探索範囲を大きくしていることが考えられる.
しかし,この手法では近傍の数も増加するため,解の探索に時間がかかりすぎてしまう.
そこで探索範囲をある程度広く保ち,探索性能を上げるために多様性の維持の手法の導入が考えられる.
鳥を減らしつつ,探索範囲が広くなることでそれぞれの解の探索時間が増加し解の精度が向上すると考えられる.
多様性を維持する手法として評価値に関わらず温度によって近傍解に解の更新を行うSAがある.MBOにSAを適用した例を付録Cに示す.
また,多様性を計る方法としてPan,Ruizらの確率分布モデル~\cite{pan2012estimation}がある.これを利用して解の更新を行う手法も考えられる.



%%%%%%%%%%%%%%%%%%%%%%%%%%%%%%%%%%%%%%%%%%%
\expandafter\ifx\csname ifdraft\endcsname\relax
  \end{document}
\fi