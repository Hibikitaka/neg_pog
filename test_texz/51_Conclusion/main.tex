\expandafter\ifx\csname ifdraft\endcsname\relax
 \documentclass{jsarticle}
 \begin{document}
\fi
%%%%%%%%%%%%%%%%%%%%%%%%%%%%%%%%%%%%%%%%%%%

\def\Path{./51_Conclusion}

%------------------------------------------------------------------------------------------------------------------  結論  ------------------------------------------------------------------------------------------------------------------%

\chapter{結論と今後の課題}

%------------------------------------------------------------------------------------------------------------------  まとめ  ------------------------------------------------------------------------------------------------------------------%
\section{結論}

将来,技術の発展により個人専用のAIエージェントの開発が予想されると考えられる.
上記のような個人専用のAIエージェントが開発された際,相手であるユーザの表情や言葉からユーザ自身の情報を得るだけではなく,心拍や脳波などの生体情報を用いた複数のメンタルの推定が可能となり定量的に数値化できれば,例えば集中とストレスの数値から課題の難易度の推定・評価やリラックスとストレスからリラックス単一よりも具体的に評価が行えるなど特定のメンタル以外の推定や推定の幅が広がることが可能であり,また表情や言葉による感情表現が苦手な人や病気などの理由から表情や言葉による感情表現が困難な人にも負担なくエージェントに情報を与えることができると考えた.
このように複数のメンタルの推定システムは将来必要であると考えられ,脳波情報を利用した複数のメンタル(集中・リラックス・ストレス)の推定システムの開発を行った.
また本システムに脳波を利用することを考えた際,装着が容易で装着者の行動を制限しないものの方が理想であると考え,本研究では簡易脳波計であるMuseSを使用した.脳波計であっても脳波の特徴があらわれるか音楽の視聴による実験を行った結果,簡易脳波計であっても脳波の特徴となるリラックスの指標である$\alpha$波や思考状態の指標である$\beta$波,ストレスの指標である$\beta/\alpha$が確認できた.
音楽視聴の実験の結果ではリラックス時には$\alpha$波の増加以外にも,$\beta$波の減少も見られた.
また実験結果から思考状態に$\beta$波が増加するだけでなく,アンケート結果で難易度が難しいと多く回答されていたものの方がストレスの指標である$\beta/\alpha$が増加していた.
上記の結果からメンタル(集中・リラックス・ストレス)を推定するための計算式を$\alpha$波や$\beta$波の強さや比率から考案し,それらが適切な推定が行えるか音楽の視聴による評価実験を行った.
評価実験の結果ではメンタルにより中央値の偏りがありメンタル同士の比較が難しかったため,実験結果である各メンタルの実数値に対し偏差値を適用した.その結果,集中・リラックス・ストレスがそれぞれ高くなるであろうと想定した実験通りに各メンタル(集中・リラックス・ストレス)が高い値を示していたことから考案した式による適切な推定が行え,メンタル推定システムの開発に成功したといえる.

%----------------------------------------------------------------------------------------------------------------  今後の課題  --------------------------------------------------------------------------------------------------------------%

\section{今後の課題}

本システムの今後の課題として3つのことを挙げる.

%-----------------------------------------------------------------------------------------------------------  各メンタルの中央値の偏り  ---------------------------------------------------------------------------------------------------------%

\subsection{各メンタルの中央値の偏り}

1つ目に各メンタルの中央値の偏りへの対応が挙げられる.
表\ref{personA}から表\ref{personE}の結果からメンタルごとに中央値の偏りが見られることがわかる.
本研究では複数のメンタルの推定を行い結果を比較することで特定のメンタル以外の推定・評価を行うことができると考えていることから,中央値の偏りへの対応は課題として挙げる3つの中で最も優先的に解決しなければいけない課題である.
具体的な対応案としては本実験で行ったように推定した実数値から偏差値の値を求める計算をシステム上で行い,求めた結果をメータに反映させることが挙げられる.
また偏差値を反映する際には,測定時に基準となる平均値を一定時間で更新する必要もある.


%--------------------------------------------------------------------------------------------------------------  推定の精度を向上  -----------------------------------------------------------------------------------------------------------%

\subsection{推定の精度を更に向上}

2つ目は推定の精度を更に向上させることである.
本研究では音楽を視聴することによって脳波のどのような特徴があらわれるか実験を行ったが,時間の都合上により実験回数自体が少なかった.
そのため動画の視聴などの別のアプローチから実験方法や実験回数を増やし多くのデータをとることができれば,各周波数帯域での特徴や周波数帯域間の傾向を見つけやすくなると考えられる.
周波数帯域間の傾向を見つけることができれば,本研究で考案した集中・リラックス・ストレスの3つの式の$\alpha$波や$\beta$波以外の新たなパラメータとして加えることができ,推定の精度の向上を見込むことができる.
また集中・リラックス・ストレスの3つの式の中ではストレスの式にのみ重みをつけた.結果ではストレスが他の2つに比べ僅かではあるが値の範囲が大きくなっているように見える.
このことから重みをつけた場合つけない場合での式の比較,重みをつけた場合どの割合での重みが最適なのかといった実験を行うことでも推定の精度を向上させる可能であると考えている.


%-----------------------------------------------------------------------------------------------------  集中・リラックス・ストレス以外のメンタルの推定  -----------------------------------------------------------------------------------------------%

\newpage
\subsection{集中・リラックス・ストレス以外のメンタルの推定}

3つ目に楽しいや怒りといった本システムで取り上げた代表的なメンタル以外のメンタルを推定するための計算式の考案が挙げられる.
本研究ではメンタルとして代表的であり,先行研究としても多く扱われている集中・リラックス・ストレスを中心に実験や推定を行うための式の考案を行った.
集中・リラックス・ストレス以外に笑いや喜び,かわいいといったメンタルに関する研究も多くはないが行われている.
それらの論文から実験方法や結果を参考にして実験を行うことで,本研究で扱った以外のメンタルを推定するための計算式の考案ができるのではないかと考えている.
また実験を行う際には参考にした論文をそのまま真似をするだけでなく,例えば音楽視聴の実験では参考にした曲のみを使用するのではなく他の曲を追加したり変更したりするなどの工夫をすることが重要である.


%%%%%%%%%%%%%%%%%%%%%%%%%%%%%%%%%%%%%%%%%%%
\expandafter\ifx\csname ifdraft\endcsname\relax
  \end{document}
\fi