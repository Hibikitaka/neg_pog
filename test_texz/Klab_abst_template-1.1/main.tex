\documentclass[twocolumn,a4paper,dvipdfmx]{klabAbst}
\usepackage[T1]{fontenc}
\usepackage{lmodern}
\usepackage{textcomp}
\usepackage{latexsym}
%\usepackage[fleqn]{amsmath}
%\usepackage{amssymb}
\usepackage{float}
\usepackage{graphicx}

%%%%%%%%%%%%%%%%%%%%%%%%
% タイトルエリアの定義
%%%%%%%%%%%%%%%%%%%%%%%%
% 発表番号:学科から指定された番号を半角英数で入力すること.
\paperid{A1-1} 
% 題目 注意:{}内先頭のの \bf は残すこと!
\title{\bf バイタルデータにもとづくメンタル推定・サポートエージェントの開発 ~脳波計を用いた脳状態の推定~}
% 学生番号,氏名
\author{片山研究室 T22J114 高渕 響} 

%%%%%%%%%%%%%%%%%%%%%%%%
% ページ数を定義
%%%%%%%%%%%%%%%%%%%%%%%%
% 学科から指定されたページ数を入れること.
\pagenum{0}


% ページ,タイトル初期化,配置(!触らないこと!)
\begin{document}
\maketitle
% ページ数を表示(!触らないこと!)
\thispagestyle{plain} % 1ページ目に番号を出すために必要
\pagestyle{plain}     % 2ページ目以降も継続


%%%%%%%%%%%%%%%%%%%%%%%%
% 本文
%%%%%%%%%%%%%%%%%%%%%%%%

\section{まえがき}
日本における不登校児童生徒数は年々増加傾向にある.
不登校に至る要因は多岐にわたるが,無気力や不安といった心理的要因を背景とするものが全体の約4割を占めている.
さらに,自身でも明確なきっかけを把握できないまま不登校となる児童生徒も約3割存在する.
この背景には,スマートフォンやSNSの急速な普及により,幼少期からインターネットに触れる機会が増加し,対面でのコミュニケーション機会が相対的に減少していることが挙げられる.
その結果,自身の感情や思考を言語化する能力が低下し,本音や不安,葛藤を周囲に適切に伝えることが困難になっている.
このような状況では,心理的ストレスや違和感が家庭や学校において認識されにくく,支援の遅れが生じる可能性が高まる.
結果として,問題が顕在化しないまま不登校に至るケースが増加していると考えられる.
そこで本研究では,脳波などのバイタルデータに基づく脳状態推定に着目し,集中,リラックス,ストレスといった状態を客観的に推定する新たな手法の提案を目的とする.


\section{脳波の概要}
脳波は周波数の帯域により名称が付けられる.表\ref{tb:eachfrequency}に名称,周波数範囲,出現が見られる心理状況を挙げる.
% 記号は原則,数式機能を用いて記述する.インライン数式(文章内に記述される数式,数式用文字)は,$で囲むことで記述できる.


% 文章ブロックと見出し,図表間で垂直余白が気になる場合,vspaceを用いて余白を調整できる.
% 以下のように負の数を指定することで,下側の要素を上方向に調整.
\vspace{-3mm}

% 図表の挿入.
% labelは,図,表,擬似コードなどで共通して使うため,タグの先頭に「fig:」「table:」「code:」など付けておくと便利.
% 図を中央揃えしたいため,centering属性を付加している.
%---------------  表挿入  ---------------%
\begin{table}[H]
\begin{center}
\normalsize
\caption{各周波数成分範囲}
\resizebox{\linewidth}{!}{
\begin{tabular}{|c|c|c|}
\hline
タイプ & 測定可能データ(Hz) & 心理状態 \\
\hline
$\delta$波 & 0.5~2.75 & 夢を見ない不快睡眠,ノンレム睡眠,無意識 \\
\hline
$\theta$波 & 3.5~6.75 & 直感的,創造的,想起,空想,幻想,夢 \\
\hline
low$\alpha$波 & 7.5~9.25 & リラックス,平穏意識的 \\
\hline
high$\alpha$波 & 10~11.75 & リラックスしているが集中している,統合的 \\
\hline
low$\beta$波 & 13~16.75 & 思考,自己および環境の認識 \\
\hline
high$\beta$波 & 18~29.75 & 警戒,動揺 \\
\hline
low$\gamma$波 & 31~39.75 & 記憶,高次精神活動 \\
\hline
mid$\gamma$波 & 41~49.75 & 視覚情報処理 \\
\hline
\end{tabular}
}
\label{tb:eachfrequency}
\end{center}
\end{table}

   %---------------  表終了  --------------%

\vspace{-3mm}

一般にリラックス状態時には$\alpha$波の振幅が大きくなる.
反対に緊張時には$\alpha$波の振幅が小さくなり$\beta$波の出現が見られるなどの特徴がある.
また$\beta$波は思考状態,$\gamma$波は高次精神活動と関連しているとされ,$\beta$/$\alpha$は脳の活動を見るための指標としてよく用いられる.

\section{脳状態を推定する各指標}

\vspace{-2mm}

\section{開発したシステムの評価と実験結果}
%---------------  表挿入  ---------------%
\vspace{-1mm}

\begin{figure}[H]
    \begin{center}
        \includegraphics[scale=0.2]{fig/Comparison_allmethods.png}
        \vspace{-1mm}
        \caption{各指標毎のΔValue値の比較図}
        \label{fig:n6-7}
    \end{center}
\end{figure}
\vspace{-2mm}
   %---------------  表終了  --------------%
\vspace{-1mm}

% 平均精度テーブル
\begin{table}[h]
  \centering
  \caption{各近傍構造を用いたLSの平均精度}
  \label{table:accuracyAvg}
  \scriptsize
  \vspace{2mm}
  \begin{tabular}{|c|c|c|c|c|}
    \hline
             & N5                            & N6                            & N7                            & N8                            \\ \hline
    Average  & \multicolumn{1}{r|}{0.347} & \multicolumn{1}{r|}{0.299} & \multicolumn{1}{r|}{0.297} & \multicolumn{1}{r|}{0.242} \\ \hline
  \end{tabular}
\end{table}

\vspace{-2mm}


\section{むすび}

%%%%%%%%%%%%%%%%%%%%%%%%%%%%%%%%%%%%%%%%%%%%%%%%%%%%%%%%%%%%%%%%%%%%%%
%bibTeXの設定(原則触らなくてよい)
\bibliographystyle{junsrt} % 日本語対応の引用順スタイル
\bibliography{ref}    % .bib ファイルの名前(拡張子抜き)
\end{document}
