\documentclass[twocolumn,a4paper,dvipdfmx]{klabAbst}
\usepackage[T1]{fontenc}
\usepackage{lmodern}
\usepackage{textcomp}
\usepackage{latexsym}
%\usepackage[fleqn]{amsmath}
%\usepackage{amssymb}
\usepackage{float}
\usepackage{graphicx}

%%%%%%%%%%%%%%%%%%%%%%%%
% タイトルエリアの定義
%%%%%%%%%%%%%%%%%%%%%%%%
% 発表番号:学科から指定された番号を半角英数で入力すること.
\paperid{A1-1} 
% 題目 注意:{}内先頭のの \bf は残すこと!
\title{\bf バイタルデータにもとづくメンタル推定・サポートエージェントの開発 ~脳波計を用いた脳状態の推定~}
% 学生番号,氏名
\author{片山研究室 T22J114 高渕 響} 

%%%%%%%%%%%%%%%%%%%%%%%%
% ページ数を定義
%%%%%%%%%%%%%%%%%%%%%%%%
% 学科から指定されたページ数を入れること.
\pagenum{0}


% ページ,タイトル初期化,配置(!触らないこと!)
\begin{document}
\maketitle
% ページ数を表示(!触らないこと!)
\thispagestyle{plain} % 1ページ目に番号を出すために必要
\pagestyle{plain}     % 2ページ目以降も継続


%%%%%%%%%%%%%%%%%%%%%%%%
% 本文
%%%%%%%%%%%%%%%%%%%%%%%%

\section{まえがき}
令和4年に行われた文部科学省の調査によると,日本における不登校児童生徒数は年々増加傾向にある.
不登校に至る要因は多岐にわたるが,無気力や不安といった心理的要因を背景とするものが全体の約4割を占めている.
この背景には,スマートフォンやSNSの普及により,幼少期からインターネットに触れる機会の増加に伴い,対面でのコミュニケーション機会が減少傾向にあることが原因だと考えられる.
その結果,自身の感情や思考を言語化する能力が低下し,本音や不安,葛藤を周囲に適切に伝えることが困難になっている.
このような状況では,心理的ストレスや違和感が家庭や学校において認識されにくく,支援の遅れが生じる可能性が高まる.
そこで本研究では,非言語情報かつメンタル状態と因果性を持つ点から,脳波などのバイタルデータを用いて脳状態の推定システムの開発を目的とする.
その中でも特に脳状態の推定を行う上で用いられる従来手法より正確なメンタル推定を可能にする提案手法の有効性を示す.

\section{脳波の概要}
脳波は周波数の帯域により名称が付けられる.表に名称,周波数範囲,出現が見られる心理状況を挙げる.
% 記号は原則,数式機能を用いて記述する.インライン数式(文章内に記述される数式,数式用文字)は,$で囲むことで記述できる.


% 文章ブロックと見出し,図表間で垂直余白が気になる場合,vspaceを用いて余白を調整できる.
% 以下のように負の数を指定することで,下側の要素を上方向に調整.
\vspace{-3mm}

% 図表の挿入.
% labelは,図,表,擬似コードなどで共通して使うため,タグの先頭に「fig:」「table:」「code:」など付けておくと便利.
% 図を中央揃えしたいため,centering属性を付加している.
%---------------  表挿入  ---------------%
\begin{table}[H]
\begin{center}
\normalsize
\caption{各周波数成分範囲}
\resizebox{\linewidth}{!}{
\begin{tabular}{|c|c|c|}
\hline
タイプ & 測定可能データ(Hz) & 心理状態 \\
\hline
low$\alpha$波 & 7.5~9.25 & リラックス,平穏意識的 \\
\hline
high$\alpha$波 & 10~11.75 & リラックスしているが集中している,統合的 \\
\hline
low$\beta$波 & 13~16.75 & 思考,自己および環境の認識 \\
\hline
high$\beta$波 & 18~29.75 & 警戒,動揺 \\
\hline
\end{tabular}
}
\label{tb:eachfrequency}
\end{center}
\end{table}

   %---------------  表終了  --------------%

\vspace{-3mm}

一般にリラックス状態時には$\alpha$波の振幅が大きくなる.
反対に緊張時には$\alpha$波の振幅が小さくなり$\beta$波の出現が見られるなどの特徴がある.
また$\beta$波は思考状態,$\beta$/$\alpha$は脳の活動を見るための指標としてよく用いられる.

\section{脳波評価手法}
脳状態を推定するための指標は数多く考案されている.Engagement Index(EI)やFrontal Alpha/Beta Asymmetry(FAA),Sample Entropy(SampEn)などが挙げられる.
しかし,それらの指標は特定の一つのアプローチにのみに有効である.平成30年の研究で提案された手法では,複数のメンタル状態を推定する手法として
集中値(CC)やリラックス値(RC),ストレス値(SC)を測定する計算式を提案している.
以下がメンタル推定式(ccrcsc)である.
\begin{eqnarray}
&&\hspace{-2.5em} CC=\min \{100, \lfloor\frac{\beta}{2}(1+\frac{1}{\alpha})\times\frac{100}{2}\rfloor\}\nonumber \\ 
&&\hspace{-2.5em} RC=\min \{100, \lfloor(\max \{0,(1.0-\frac{\beta}{3})\}+\frac{\alpha}{2})\times\frac{100}{2}\rfloor\} \nonumber \\
&&\hspace{-2.5em} SC=\min \{100,\lfloor(\max \{0,\frac{1.0-\frac{\alpha}{3}}{5}\}+(\frac{\frac{\beta}{2\alpha}\times4}{5}))\times100 \rfloor\} \nonumber
\end{eqnarray}
\vspace{-1mm}
この式はEI,FAA,SampEnと比較すると,より高い精度で脳状態の推定が行うことができる.平成30年の研究の課題として他のパラメータを加えたり,重みの値の調整によって更に推定の精度を向上が臨めると記述されていた.
そのため,本研究では各メンタル状態をこの計算式のパラメータを調整したccrcsc\_optimizeを提案する.以下がパラメータ改善式である.
\begin{eqnarray}
&&\hspace{-2.5em} CC=\min \{100, \lfloor\frac{\beta}{2}(1+\frac{1}{\alpha+1.2})\times{60}\rfloor\}\nonumber \\ 
&&\hspace{-2.5em} RC=\min \{100, \lfloor(\max \{0,(1.0-\frac{\beta}{3})\}+\frac{\alpha}{1.5})\times{60}\rfloor\}  \nonumber \\
&&\hspace{-2.5em} SC=\min \{100,\lfloor(\max \{0,\frac{1.0-\frac{\alpha}{3}}{5}\}+(\frac{\frac{\beta}{2\alpha}\times4.2}{5}))\times100 \rfloor\} \nonumber
\end{eqnarray}
\vspace{-1mm}

調整後のCC式では,分母に定数項($+1.2$)を加えることで,$\alpha$帯域が低い場合でも値の発散を抑制し,課題区間における$\beta$帯域増加のみをより安定して反映するように調整した.\\
調整後のRC式では,$\alpha$帯域の寄与を $1.5$ で倍率を下げることで,個人差の影響を低減し,課題遂行に伴う$\beta$帯域変化の影響をより明確に反映するように調整した.\\
調整後のSC式では,$\frac{\beta}{2\alpha}$項の倍率を $4.2$ に増加させることで,課題遂行時に生じる覚醒・緊張状態に対する感度を増加するように調整した.\\
\vspace{-2mm}
\section{実験結果}
提案手法の有効性を示すために,実験を行った.
被験者5人にそれぞれ集中状態,リラックス状態,ストレス状態を誘発する3種類の音源の視聴を行い,各CC,RC,SCの指標の比較を行った.
それぞれ各指標を比較するための値として平常時と測定時の差を示すΔValue値を用いる.
実験結果を図1に示す.

%---------------  表挿入  ---------------%
\vspace{-1mm}

\begin{figure}[H]
    \begin{center}
        \includegraphics[scale=0.3]{fig/Comparison_allmethods.png}
        \vspace{-1mm}
        \caption{各指標毎のΔValue値の比較図}
        \label{fig:n6-7}
    \end{center}
\end{figure}
\vspace{-2mm}
   %---------------  表終了  --------------%
図1の結果から,集中値はSampEnが,リラックス値はccrcsc\_optimize,ストレス値はEIが最も高い値が表れた.
Δvalue値が高いとその脳状態の実態を正しく出力出来ていることを示している.
しかしSampEnやEIはその他の脳状態において負の値が大きく表れていた.
ccrcsc\_optimizeは各指標のΔValueにおいて高い値が表れた.
そのため考案したccrcsc\_optimizeが従来手法と比較して高い水準で脳状態の判定が行えているといえる.

\vspace{-2mm}

\section{むすび}

%%%%%%%%%%%%%%%%%%%%%%%%%%%%%%%%%%%%%%%%%%%%%%%%%%%%%%%%%%%%%%%%%%%%%%
%bibTeXの設定(原則触らなくてよい)
\bibliographystyle{junsrt} % 日本語対応の引用順スタイル
\bibliography{ref}    % .bib ファイルの名前(拡張子抜き)
\end{document}
