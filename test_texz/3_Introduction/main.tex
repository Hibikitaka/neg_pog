\expandafter\ifx\csname ifdraft\endcsname\relax
\documentclass[a4paper,11pt, dvipdfmx]{jreport}
\usepackage[dvipdfmx]{graphicx}
\usepackage{here}
 \begin{document}
\fi
%%%%%%%%%%%%%%%%%%%%%%%%%%%%%%%%%%%%%%%%%%%

\def\Path{./3_Introduction}

%%%%%%%%%%%%%%%%%%%
\chapter{序 論}
%%%%%%%%%%%%%%%%%%%

%%%%%%%%%%%%%%%%%%%%%%%%%%%%%%%%%%%%%%%%
\section{研究の背景と目的 \label {sec:context}}
日本において不登校児童.生徒の数は年々増加傾向にあり,深刻な社会問題として広く認識されている.

\begin{figure}[H]
    \centering
        \includegraphics[width=103mm]{\Path /Image/graph1-1.png}
        \caption{不登校児童生徒数の推移}
\end{figure}

図1.1の令和4年に行われた文部科学省の調査によると,不登校児童生徒数はH30年度以前は年間15万人だったが,H30年以降を機に増加し続け,R4年度には以前の約2倍以上に増えている.
不登校は一時的な個人の問題に留まらず,長期的には学習機会の損失や社会的孤立,さらには将来的な就労や社会参加にも影響を及ぼす可能性があることが指摘されている.
このような状況を受け,不登校の予防や早期発見,支援の在り方について,多方面からの研究および実践的取り組みが進められている.

\begin{figure}[H]
    \centering
        \includegraphics[width=150mm]{\Path /Image/graph1-2.png}
        \caption{不登校の原因の内訳}
\end{figure}

\begin{figure}[H]
    \centering
        \includegraphics[width=150mm]{\Path /Image/graph1-3.png}
        \caption{児童生徒が感じた不登校になったきっかけ}
\end{figure}

不登校の原因は多岐に渡るが図1.2で示しているように,無気力や不安な感情に陥ったため不登校になった児童が全体の4割を占めている.
また不登校児童が増えている理由として自分でもきっかけがよく分からずに不登校になった児童が約3割存在していることが図1.3で示されている.
このような背景にあるのは,児童を取り巻くコミュニケーション環境の変化である.スマートフォンやSNSの急速な普及により,若いうちからインターネットに触れる機会が大幅に増えた.
遠くの人と気軽にコミュニケーションが取れる利点もある一方で,対面でのコミュニケーション機会の減少をはじめ,自身の感情や思考を言語化する能力の低下を招いている.
また,インターネット上では自身の本音や不安,葛藤を直接表に出すことを避け,表面を取り繕った自己表現を行う傾向が強まっているとされる.
以上のような状況下では,本人が抱える心理的ストレスや違和感が周囲に伝わりにくく,家庭や学校においても児童の抱えている異変に気付くことが遅れてしまう可能性がある.
これらの要因より保護者や教師が児童の異変に気付くのに遅れ、結果的に不登校に繋がっていると考えられている.
不登校への対応は,本人や保護者,教師による対話,観察に大きく依存している.そのため,本人が自身の心理状態を伝えるのを避ける場合,こうした従来の対処方法だけでは限界がある.
以上の背景から,主観的な訴えや表情,言動だけに頼らず,より客観的な指標を用いて個人の心理状態を把握する手法の必要性が高まっている.

そこで注目したのが,人間の生体情報,即ちバイタルデータを活用したメンタル状態の推定である.
脳波や心拍,呼吸といったバイタルデータは,自律神経活動や脳の状態と密接に関連しており,ストレスや不安,集中,リラックスといった心理状態の変化を反映している.
これらのデータは,本人の意識的な行動や言語表現を必要とせずに取得できる大きな利点を持つ.
センサ技術やウェアラブルデバイスの発展により,脳波計,心拍センサ,呼吸センサなどを比較的容易に利用できるようになった.
これに伴い,個人向けの感情認識ロボットやAIエージェント,メンタルヘルス支援アプリケーションなども普及し始めている.これらのシステムの多くは,ユーザとの対話データや簡易的な生体情報を元に,感情推定やストレス評価を行うことを目的としている.
更に,機械学習や深層学習技術の進展により,多次元なバイタルデータから特徴量を抽出して,心理状態のパターンを学習することが可能である.これにより,単一の指標では捉えきれなかった複雑なメンタル状態を,一定の精度で推定が行える可能性が示されている.

本研究では,自律的な介入を行うAIエージェントシステムの開発を行う上で必要な集中状態やリラックス状態、ストレス状態といった脳状態の推定を行う.
これまで脳波データを元に脳状態の推定を行うための理論や手法は数多く考案されてきた.しかし、それらのほとんどが脳状態の推定おいてスタンダードな手法として確立されていない.
そのため、本研究では、これまで考案された理論や手法をもとに、新たな脳状態推定手法の提案を行う.
更にメンタル状態の推定だけでなく,その推定結果に基づいてAIエージェントが人間側へ自律的に働きかけを行うことを目指す.具体例として,ストレスや不安が高まっていると推定された場合には,安心感を与えるメッセージを提示したり,リラックスを促す行動を提案したりするなど,ユーザの状態に応じた適切な介入を行うことなどが挙げられる.
本研究によって,人間が自身の状態を言語化できない場合や,周囲が異変に気付きにくい状況においても,客観的なバイタルデータを通じてメンタル状態を把握し,早期の支援に繋がることが期待される.特に,不登校の予防や初期段階での支援において,本研究の成果が新たなアプローチを提供するものと考えている.

\newpage
\section{本論文の構成}
本論文の構成を以下に示す.
\begin{description}
    \item[第2章] 第2章では,本研究で扱う脳波の基礎的な概念について説明する.具体的には,脳波の定義や発生原理,周波数帯ごとの特徴,およびメンタル状態との関係について概説する.
    \item[第3章] 第3章では,本研究において脳波と脳状態の関連性を求めるために参考にした理論や手法の考え方について述べる.
    \item[第4章] 第4章では,本研究で使用する脳波計測装置であるMuse Sについて説明する.装置の構成,計測可能なデータの種類,および本研究における計測環境について述べる.
    \item[第5章] 第5章では,考案したシステムの評価実験の方法,実験条件,および得られた結果について述べる.
    %\item[第6章] 第6章では, する.
    最後に第6章で結論と課題・今後の検討事項について述べる.
\end{description}
\expandafter\ifx\csname ifdraft\endcsname\relax
  \end{document}
\fi