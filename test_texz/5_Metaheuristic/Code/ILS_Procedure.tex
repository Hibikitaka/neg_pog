\expandafter\ifx\csname ifdraft\endcsname\relax
% \documentclass{jsarticle}
% \begin{document}
\fi
%%%%%%%%%%%%%%%%%%%%%%%%%%%%%%%%%%%%%%%%%%%

\begin{figure}[h]
\normalsize
\baselineskip = 0.05pt 
\sfcode`;=3000  \def\q{\hspace*{2mm}}
\setbox0\vbox{ %]
\begin{quote}
\begin{list}{}{\rightmargin=15mm \leftmargin=10mm \labelwidth=20mm %% \parsep = -.7mm
} 
\item[  {\bf 反復局所探索法}] 
\item[ステップ1] 適当な解を初期解として局所探索法を行い,局所最適解$x$を得る.
\item[ステップ2] $x$に対してランダムな解の変更を加え,$x'$を得る.
\item[ステップ3] $x'$を初期解として局所探索法を行い,局所最適解$x$を得る.
\item[ステップ4] $f(x') < f(x)$ならば$x := x'$とする.
\item[ステップ5] 終了条件を満たせば,$x$を出力して探索を終了する.
\item[         ] そうでなければ,ステップ2に戻る.
\end{list} 
\end{quote}
}
\begin{center}\fbox{\box0}\end{center}
\caption{反復局所探索法の手続き~\cite{ruiz2007simple}} \label{fig:ils-basic}
\end{figure}

%%%%%%%%%%%%%%%%%%%%%%%%%%%%%%%%%%%%%%%%%%%
\expandafter\ifx\csname ifdraft\endcsname\relax
%  \end{document}
\fi