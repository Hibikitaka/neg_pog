\expandafter\ifx\csname ifdraft\endcsname\relax
 \documentclass{jsarticle}
 \begin{document}
\fi
%%%%%%%%%%%%%%%%%%%%%%%%%%%%%%%%%%%%%%%%%%%

\def\Path{./5_Metaheuristic}

%%%%%%%%%%%%%%%%%%%%%%%%%%%%
\chapter{脳波データに関する理論と手法}

%%%%%%%%%%%%%%%%%%%%%%%%%%%
本章では,まず,組合せ最適化問題に対するメタ戦略について説明する.
次に,メタ戦略の基本戦略である局所探索法について述べる.
まず,代表的なメタ戦略の例として反復局所探索法について述べ,我々が改良する元のアルゴリズムであるMBOについて説明する

%%%%%%%%%%%%%%%%%%%%%%%%%%%%


\section{Engagement Index(EI)に基づく脳波指標}

\subsection{Engagement Indexの概念}

Engagement Index(以下,EI)は,脳波(EEG)信号に含まれる周波数帯域成分の比率を用いて,
被験者の\textbf{注意集中度}や\textbf{課題関与度}を定量化するために提案された指標である.
EIは,主に$\alpha$波,$\beta$波,$\theta$波の相対的なパワー分布に基づいて算出される.

従来研究において,$\beta$波は認知的負荷や能動的思考と関連し,
$\alpha$波および$\theta$波はリラックス状態や注意低下と関係することが報告されている.
この知見に基づき,EIは「覚醒・集中を促進する成分」と
「非集中状態を示す成分」の比として定義される.

\subsection*{認知的負荷と脳波指標}

認知的負荷(cognitive load)とは,
課題遂行時にワーキングメモリや注意資源が
どの程度消費されているかを表す概念である.
認知的負荷が高い状態では,
外部刺激の処理や内部情報の保持・操作が増加し,
前頭部を中心とした脳活動が活性化することが知られている.

脳波においては,
β帯域の増加および
α帯域の抑制が
認知的負荷の上昇と関連づけられており,
これらの周波数特性を組み合わせた指標が
課題集中度の評価に用いられてきた.


\subsection{Engagement Indexの定義式}

本研究では,Engagement Indexを以下の式で定義する.

\begin{equation}
EI = \frac{P_{\beta}}{P_{\alpha} + P_{\theta} + \varepsilon}
\end{equation}

ここで,
$P_{\alpha}$,$P_{\beta}$,$P_{\theta}$ はそれぞれ
$\alpha$帯(8--13 Hz),$\beta$帯(13--30 Hz),$\theta$帯(4--8 Hz)
における脳波パワーを表す.
また,$\varepsilon$は分母が0となることを防ぐための微小定数である.

この式は,$\beta$波が優勢であるほどEIが高くなり,
逆に$\alpha$波や$\theta$波が優勢な場合にはEIが低下する構造を持つ.

\subsection{心理量としての解釈}

EIは絶対的な心理状態を直接表すものではなく,
\textbf{同一被験者内での相対的変化量}として解釈される指標である.
特に,安静時(baseline)と課題遂行時のEI差分を用いることで,
課題への関与度の変化を定量的に評価することが可能となる.

本研究では,以下の差分量を用いて解析を行う.

\section{Frontal Alpha/Beta Asymmetry(FAA派生指標)}

\subsection{Frontal Asymmetryの概念}

前頭部脳波の左右非対称性(Frontal Asymmetry)は,
感情処理,動機づけ,注意制御などの心理状態と密接に関係することが
示されている。
特に前頭部$\alpha$波の左右差は,
接近動機と回避動機の指標として広く用いられてきた.

一般に,$\alpha$波パワーは皮質活動の抑制度合いを反映すると考えられており,
$\alpha$波パワーが低いほど局所的な神経活動が高い状態を示す.
そのため,左右前頭部の$\alpha$波パワー差を用いることで,
機能的な活動非対称性を定量化することが可能となる.

\subsection*{接近動機と回避動機}

接近動機(approach motivation)とは,
報酬獲得や目標達成といった
正の結果に向かって行動を選択しようとする動機づけを指す.
一方,
回避動機(avoidance motivation)は,
罰や失敗, 不快刺激を回避することを目的とした
行動選択に関与する動機づけである.

接近動機は主に左前頭部の活動増加と関連し,
回避動機は右前頭部の活動増加と関連すると報告されている.
脳波研究では,
前頭部α帯域のパワー低下は局所的な皮質活動の亢進を反映すると考えられており,
左右前頭部α帯域の非対称性は,
接近・回避傾向を推定する指標として広く用いられてきた.

このような前頭部活動の側性は,
単なる情動価の違いにとどまらず,
課題に対する能動的関与や行動選択の方向性とも関連する.
したがって,
本研究で用いるFrontal Alpha/Beta Asymmetry派生指標は,
被験者が課題に対して
接近的に関与しているか,
あるいは回避的に処理しているかを
反映する可能性がある.

\subsection{Frontal Alpha Asymmetry(FAA)の定義}

Frontal Alpha Asymmetry(FAA)は,
左右前頭部電極(本研究ではAF7およびAF8)における
$\alpha$帯域パワーの比を用いて次式で定義される.

\begin{equation}
FAA_{\alpha} = \log \left( \frac{P_{\alpha}^{\text{right}}}{P_{\alpha}^{\text{left}}} \right)
\end{equation}

ここで,
$P_{\alpha}^{\text{left}}$ および $P_{\alpha}^{\text{right}}$ は,
それぞれ左前頭部(AF7),右前頭部(AF8)における
$\alpha$帯域パワーを表す.

FAAが正の値をとる場合,
右前頭部よりも左前頭部の皮質活動が高い状態を示し,
接近的・積極的な心理状態と関連すると解釈される.

\subsection{Alpha/Beta Asymmetryへの拡張}

本研究では,従来の$\alpha$波のみに基づくFAAに加え,
$\beta$波成分を考慮したFrontal Alpha/Beta Asymmetry指標を導入する.

$\beta$波は注意集中や課題遂行時の認知負荷と関連することから,
$\alpha$波と$\beta$波の比を用いることで,
覚醒度と抑制状態を同時に反映した指標の構築が可能となる.

本研究で用いるAlpha/Beta Asymmetry指標は次式で定義される.

\begin{equation}
FAA_{\alpha/\beta}
=
\log \left(
\frac{P_{\alpha}^{\text{right}} / P_{\beta}^{\text{right}}}
     {P_{\alpha}^{\text{left}}  / P_{\beta}^{\text{left}}}
\right)
\end{equation}

この式は,
左右前頭部における$\alpha/\beta$比の非対称性を表しており,
$\beta$波優勢な活動状態を強調する構造を持つ.

\subsection{心理量としての解釈}

$FAA_{\alpha/\beta}$ は絶対値よりも,
安静時と課題時の変化量に基づいて解釈される指標である.
本研究では,以下の差分量を用いる.

\begin{equation}
\Delta FAA_{\alpha/\beta}
=
FAA_{\alpha/\beta}^{\text{task}}
-
FAA_{\alpha/\beta}^{\text{baseline}}
\end{equation}

$\Delta FAA_{\alpha/\beta} > 0$ は,
課題遂行時に左前頭部の相対的活動が増加したことを示し,
能動的・接近的な認知状態への移行を示唆する.

\subsection{本研究における位置づけ}

Frontal Alpha/Beta Asymmetry指標は,
左右非対称性という空間的情報を利用する点において,
周波数比に基づくEngagement Indexとは異なる特徴を持つ.

\section{Sample Entropyに基づく複雑度指標}

\subsection{脳波の複雑性と非線形指標}

脳波信号は非定常かつ非線形な時系列データであり,
単純な振幅や周波数成分のみでは,
認知状態の変化を十分に捉えられない場合がある.
特に注意制御や課題遂行時には,
信号の規則性や不規則性といった「複雑性」が
重要な情報を含むと考えられている.

このような背景から,
時系列の自己相似性に基づく非線形指標として
Sample Entropy(SampEn)が広く用いられている.

\subsection{Sample Entropyの定義}

Sample Entropyは,
時系列データにおいて
長さ$m$の部分系列が互いに類似している確率と,
長さ$m+1$の部分系列が類似している確率の比を用いて定義される.

時系列 $\{x_1, x_2, \dots, x_N\}$ に対し,
許容誤差$r$を用いると,
Sample Entropyは次式で表される.

\begin{equation}
SampEn(m,r)
=
- \log \left( \frac{A}{B} \right)
\end{equation}

ここで,
$B$は長さ$m$の部分系列が互いに類似する組の数,
$A$は長さ$m+1$の部分系列が類似する組の数を表す.

SampEnの値が大きいほど,
時系列の規則性が低く,
複雑性が高い状態であることを示す.

\subsection{CC・RC・SCによる分解表現}

Sample Entropyは,
対数変換後の単一スカラー値として扱われることが多いが,
本研究では,
複雑性の変化過程をより詳細に捉えるため,
Sample Entropyの構成要素である
$B$および$A$に対応する量を明示的に用いる.

本研究では以下の量を定義する.
\begin{itemize}
  \item $CC$(Correlation Count):
  長さ$m$の部分系列が類似する組の数
  \item $RC$(Refined Count):
  長さ$m+1$の部分系列が類似する組の数
  \item $SC$(Similarity Coefficient):
  \begin{equation}
  SC = \frac{RC}{CC}
  \end{equation}
\end{itemize}

$SC$は,
長さ$m$で類似していた構造が,
1次元拡張後も類似を保つ確率を表しており,
信号構造の「持続性」を示す指標と解釈できる.

\subsection{複雑度指標としての解釈}

$CC$は,
時系列内に存在する類似構造の総量を反映し,
信号の冗長性や周期性の強さを示す.

$RC$は,
より厳しい条件下で保持される構造の量を表し,
高次元における規則性の残存度を反映する.

$SC$は,
構造の保存率を表す確率的指標であり,
以下の関係が成り立つ.

\begin{equation}
SampEn = -\log(SC)
\end{equation}

したがって,
$SC$が小さいほど複雑性が高く,
$SC$が大きいほど規則性が強い状態を示す.

本研究では,
対数変換されたSampEn値ではなく,
$CC$,$RC$,$SC$の変化を
それぞれ独立した複雑度指標として扱う.

\subsection{心理状態評価への応用}

複雑度指標は,
覚醒度や感情価といった一次元的心理量ではなく,
認知処理の柔軟性や情報統合状態を反映する指標として解釈される.

本研究では,
安静時と課題遂行時の差分量を用い,
以下の指標を算出する.

\begin{equation}
\Delta CC = CC^{\text{task}} - CC^{\text{baseline}}
\end{equation}

\begin{equation}
\Delta RC = RC^{\text{task}} - RC^{\text{baseline}}
\end{equation}

\begin{equation}
\Delta SC = SC^{\text{task}} - SC^{\text{baseline}}
\end{equation}

これらの差分は,
課題遂行に伴う脳波構造の
規則性・持続性・複雑性の変化を表すものである.


 \chapter{脳波計測装置 Muse S}
本章では本研究で使用するMuse Sと呼ばれる脳波計測装置の概要.及び操作方法の説明を行う.

%---------------------------------------------------------------------------------------------------------------- Muse S の概要  --------------------------------------------------------------------------------------------------------------%
\section{Muse S の概要}
本研究では, 被験者のメンタル状態を推定するための生体情報として脳波を用いる. 
従来, 脳波計測には医療用 EEG 装置が用いられることが多く, 高精度な計測が可能である一方で, 装着の煩雑さや高コスト, 実験環境の制約といった課題が存在していた. これらの課題を踏まえ, 一般消費者向けの簡易型脳波計測装置が開発されている.
本研究では, 比較的高い計測精度と装着性を兼ね備えたヘッドバンド型脳波計測装置である Muse S を使用する.

%---------------------------------------------------------------------------------------------------------------- Muse S の基本構成  --------------------------------------------------------------------------------------------------------------%
\section{Muse S の基本構成}
Muse S は, カナダの InteraXon 社によって開発されたヘッドバンド型の脳波計測装置である. 本装置は, 前頭部および左右側頭部に配置された複数の電極を用いて脳波を計測する構成となっている.
Muse S の外観を図4.1に示す.
図4.1 Muse S の外観
本装置は軽量であり, 長時間装着しても被験者の負担が小さい設計となっている. また, ドライ電極を採用しているため, 電極ペーストやジェルを使用する必要がなく, 簡便に脳波計測を行うことが可能である.

\begin{figure}
	\centering
	\includegraphics[width=15cm]{\Path /Image/bird.PNG}
	\caption{V字型編隊のイメージ}
	\label{formation}
\end{figure}

\begin{figure}
	\centering
	\includegraphics[width=15cm]{\Path /Image/muses.eps}
	\caption{V字型編隊のイメージ}
	\label{formation}
\end{figure}

%---------------------------------------------------------------------------------------------------------------- Muse S の主な機能  --------------------------------------------------------------------------------------------------------------%
\section{Muse S の主な機能}
Muse S は, 脳波計測を中心とした複数の機能を備えたウェアラブルデバイスである.
第一に, 脳波のリアルタイム計測機能が挙げられる. Muse S は, $\delta$波, $\theta$ 波, $\alpha$ 波, $\beta$ 波といった周波数帯域別の脳波成分を抽出し, 一定周期で外部デバイスへ送信することが可能である. これにより, 被験者の覚醒状態や集中状態, リラックス状態の変化を即時に把握できる.
第二に, 生脳波データの取得機能である. 周波数帯域別データに加え, 電極ごとの生 EEG 波形を取得できるため, 後処理による特徴量抽出や解析にも対応可能である.
第三に, 無線通信機能である. Muse S は Bluetooth Low Energy(BLE) を用いた通信方式を採用しており, PC やスマートフォンとワイヤレスで接続できる.

%----------------------------------------------------------------------------------------------------------------  計測可能な脳波データ  --------------------------------------------------------------------------------------------------------------%
\section{計測可能な脳波データ}
Muse S では, 主に以下の周波数帯域に分類された脳波データを取得することが可能である.$\delta$ 波,$\theta$ 波,$\alpha$ 波,$\beta$ 波
これらの周波数帯域は, 被験者の精神状態や覚醒度と密接な関係を有している. $\alpha$ 波は安静時やリラックス状態で増加する傾向があり, $\beta$ 波は集中や緊張状態において顕著となる. $\theta$ 波は瞑想状態や浅い睡眠時に, $\delta$ 波は深い睡眠時に主に観測される.
本研究では, これらの周波数帯域ごとのパワー値および比率を特徴量として利用する.

\section{Muse S の装着方法}
Muse S はヘッドバンド型の構造を採用しており, 比較的容易に装着することが可能である. 正確な脳波計測を行うためには, 電極が適切な位置で皮膚と接触していることが重要となる.
装着時には, 前頭部の電極が左右の眉毛上付近に位置するように調整し, 側頭部の電極が耳の上付近に密着するように装着する. Muse S の装着例を図4.2に示す.

図4.2 Muse S の装着例

ドライ電極を使用しているため, 装着準備に要する時間は短いが, 電極の接触状態が不十分な場合, ノイズの混入や信号品質の低下が生じる可能性がある. そのため, 計測前には信号品質を確認することが重要である.

%---------------------------------------------------------------------------------------------------------------- Bluetooth 接続方法と BLED112 ドングル  --------------------------------------------------------------------------------------------------------------%
\section{Bluetooth 接続方法と BLED112 ドングル}
Muse S は Bluetooth Low Energy(BLE) による無線通信を用いて外部デバイスと接続する. PC 上で安定したデータ取得を行うため, 本研究では BLE 通信専用の USB ドングルである BLED112 を使用した.
Muse S と PC の接続構成を図4.3に示す.

図4.3 Muse S と PC の Bluetooth 接続構成

接続手順としては, まず PC の USB ポートに BLED112 ドングルを接続する. 次に Muse S の電源を入れ, ペアリング可能な状態に設定する. 専用のストリーミング用のプログラムを起動し, 接続可能な BLE デバイスの検索を行った後, Muse S を選択して接続を行う.
接続が成功すると, Muse S から送信される脳波データがリアルタイムで PC に受信される.

%---------------------------------------------------------------------------------------------------------------- データ取得および保存方法  --------------------------------------------------------------------------------------------------------------%
\section{データ取得および保存方法}
取得した脳波データは, 時系列データとして PC 上に保存される. 本研究では, 一定時間ごとにデータを区切り, 周波数帯域別の平均値や比率を算出することで, メンタル状態推定に用いる特徴量を生成した.
この構成により, 被験者の状態変化をリアルタイムに追跡しつつ, 後処理による解析も可能となる.

%---------------------------------------------------------------------------------------------------------------- 本研究における Muse S の位置づけ  --------------------------------------------------------------------------------------------------------------%
\section{本研究における Muse S の位置づけ}
本研究において Muse S は, 被験者の内的状態を客観的に取得するための主要なセンサとして位置づけられる. 自己申告やテキスト入力のみに依存せず, 脳波のバイタルデータを用いることで, 潜在的なメンタル状態を推定できる点が本研究の特徴である.
また, 装着負担が小さい Muse S は, 長時間計測や日常環境での利用にも適しており, 将来的な実用システムへの応用可能性も高い.


\chapter{考案システムの評価実験}
%%%%%%%%%%%%%%%%%%%%%%%%%%%%%%%%%%%%%%%%%%


%%%%%%%%%%%%%%%%%%%%%%%%%%%%%%%%%%%%%%%%%%%
\expandafter\ifx\csname ifdraft\endcsname\relax
  \end{document}
\fi
