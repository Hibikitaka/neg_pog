\expandafter\ifx\csname ifdraft\endcsname\relax
 \documentclass{jsarticle}
 \begin{document}
\fi
%%%%%%%%%%%%%%%%%%%%%%%%%%%%%%%%%%%%%%%%%%%

\begin{center} {\bf 要 旨} \end{center}

日本における不登校児童生徒数は年々増加傾向にあり,深刻な社会問題として広く認識されている.児童や生徒が不登校になった要因は多岐に渡るが,無気力や不安な感情に陥ったため不登校になった児童が全体の4割を占めている.
また,不登校児童が増えている理由として自分でもきっかけがよく分からずに不登校になった児童が約3割存在している.このような背景にあるのは,スマートフォンやSNSの急速な普及により,若いうちからインターネットに触れる機会が大幅に増えた.
インターネット上のコミュニケーションがの機会が増えた一方で,対面でのコミュニケーション機会が減少したことで,自身の感情や思考を言語化する能力の低下を招いている.また,インターネット上では自身の本音や不安,葛藤を直接表に出すことを避け,表面を取り繕った自己表現を行う傾向が強まっているとされる.
以上のような状況下では,本人が抱える心理的ストレスや違和感が周囲に伝わりにくく,家庭や学校においても児童の抱えている異変に気付くことが遅れてしまう可能性があるため,結果的に不登校に繋がっていると考えられる.
そこで注目したのが,人間の生体情報,即ちバイタルデータを活用したメンタル状態の推定である.脳波や心拍,呼吸といったバイタルデータは,自律神経活動や脳の状態と密接に関連しており,ストレスや不安,集中,リラックスといった心理状態の変化を反映している.
これらのデータは,本人の意識的な行動や言語表現を必要とせずに取得できる大きな利点を持つ.
そこで本研究では,自律的な介入を行うAIエージェントシステムの開発を行う上で必要な集中状態やリラックス状態,ストレス状態といった脳状態の推定を行う.その中でも特に脳状態の推定の理論や手法に着目し,新たな脳状態推定手法の提案を行う.

\begin{center} {\bf キーワード} \end{center} \vspace{0.01em}
精神状態,脳波,Muse S, AIエージェント


%%%%%%%%%%%%%%%%%%%%%%%%%%%%%%%%%%%%%%%%%%%.
\expandafter\ifx\csname ifdraft\endcsname\relax
  \end{document}
\fi