第3章 言語モデルの概要と本研究への適用
3.1 自然言語処理と言語モデルの概要
自然言語処理(Natural Language Processing. NLP)とは, 人間が日常的に使用する自然言語を対象として, その意味や構造を計算機によって解析.理解.生成することを目的とした研究分野である. 自然言語は, 文法的に曖昧であるだけでなく, 文脈や話者の意図, 感情など多くの非明示的情報を含むため, 計算機による処理が本質的に困難であるという特徴を持つ.
従来の自然言語処理では, 文法規則や単語辞書を用いたルールベースの手法が主流であった. しかし, これらの手法は, 設計者の想定外の表現や例外に弱く, 大規模な実運用には限界があった. その後, 統計的手法や機械学習手法が導入され, 大量のデータから言語の傾向を学習するアプローチが一般化した.
近年では, 深層学習の発展により, 言語モデルを中心とした自然言語処理技術が急速に進歩している. 言語モデルは, 単語や文の出現確率を学習し, 文脈に基づいて文章の意味理解や生成を行うモデルである. 言語モデルを用いることで, 単語単体ではなく, 文全体や文章間の関係性を考慮した処理が可能となる.
本研究では, 被験者が入力する文章から心理的状態を推定し, さらにその結果をもとに言語的な介入を行うAIエージェントシステムの構築を目的としている. そのため, 言語モデルは本研究における中核技術の一つであり, 高速性.安定性.柔軟性といった観点から複数の言語モデルを組み合わせて利用する.
3.2 辞書形式による感情判定手法の詳細
辞書形式による感情判定手法は, 自然言語処理において古くから用いられてきた手法の一つである. この手法では, 各単語に対してあらかじめ感情極性や評価値を割り当てた感情辞書を用い, 文章中に含まれる単語の極性情報を集計することで, 文章全体の感情傾向を推定する.
例えば, ポジティブな意味を持つ単語が多く含まれる文章は肯定的であると判断され, ネガティブな単語が多く含まれる文章は否定的であると判断される. この処理は非常に単純であり, 計算コストが低く, モデル学習を必要としないという利点を持つ.
また, 辞書形式の手法は, 判定結果の根拠が明確であるという特徴を持つ. どの単語がどのような影響を与えたのかを人間が容易に確認できるため, 結果の解釈性が高い. この点は, 医療や教育など説明責任が求められる分野において重要な要素となる.
一方で, 辞書形式には多くの課題も存在する. 第一に, 辞書に登録されていない単語や新語に対応できない点が挙げられる. 第二に, 否定表現や文脈による意味変化を考慮できない点である. 例えば, 「良くない」という表現は否定的であるにもかかわらず, 単語単位では誤った評価が行われる可能性がある.
本研究では, これらの課題を明確にするため, 辞書形式による感情判定を基準手法として実装し, 後述する機械学習ベースの手法との比較対象とした.
3.3 fastTextの基本概念と技術的背景
fastTextは, Facebook AI Researchによって提案された自然言語処理モデルであり, 高速かつ高精度なテキスト分類を目的として開発された. fastTextは, 単語を文字N-gramに分解して特徴量として扱うという独自の手法を採用している.
従来の単語ベースのモデルでは, 辞書に存在しない単語や表記揺れに弱いという問題があった. fastTextでは, 単語内部の文字列情報を利用することで, 未知語に対しても意味的特徴をある程度推定できる. この特性は,日本語のように表記の多様性が高い言語において特に有効である.
またfastTextは, ニューラルネットワークを用いながらも, 非常に単純な構造を持つ. そのため, 学習および推論が高速であり, 大規模データを扱う場合やリアルタイム処理が求められる場面でも高い実用性を持つ.
3.4 fastTextの学習手法と本研究での適用
fastTextは教師あり学習によりテキスト分類モデルとして学習される. 学習データは, 入力文章とそれに対応する感情ラベルから構成される. 学習時には, 文章中の単語および文字N-gramをベクトル表現に変換し, それらの平均ベクトルを用いて分類が行われる.
この学習方式により, fastTextは文章全体の傾向を効率的に捉えることが可能である. また, モデル構造が単純であるため, 過学習が起こりにくく, 少量データでも安定した性能を示す傾向がある.
本研究では, fastTextを用いて入力文章のネガティブ・ポジティブ判定を行い, その結果を数値的な指標として扱う. この指標は, バイタルデータと統合され, メンタル状態推定の重要な要素として利用される.
3.5 F1スコアによる評価指標の詳細な説明
感情判定手法の性能評価には, F1スコアを用いた. F1スコアは, 適合率と再現率の調和平均として定義される指標であり, 分類性能を総合的に評価することができる.
感情分析タスクでは, ポジティブとネガティブの出現頻度に偏りが存在する場合が多い. このような状況では, 単純な正解率では性能を正しく評価できない. F1スコアは, 誤判定の影響を考慮した評価が可能であり, 実運用を想定した指標として適している.
本研究では, 辞書形式およびfastText形式の両手法についてF1スコアを算出し, その推移を比較することで, 手法の有効性を定量的に評価した.
3.6 性能比較結果
本節では, 本研究で実施したネガティブ・ポジティブ感情判定における性能比較結果について述べる.
 本研究では, 辞書形式による感情判定手法と, fastText を用いた機械学習ベースの感情判定手法の2種類を実装し, それぞれの性能を比較した.
感情判定タスクにおいては, 単純な正解率のみではクラスの偏りによる評価の歪みが生じる可能性がある. そのため本研究では, Precision と Recall の調和平均である F1 スコアを評価指標として採用した. F1 スコアは, 偽陽性および偽陰性の双方を考慮した指標であり, 感情判定のような二値分類問題において広く用いられている.
辞書形式の感情判定では, あらかじめ定義された感情語辞書を用い, 入力文中に含まれる単語の極性スコアを集計することで最終的な判定を行った. この手法は実装が容易で計算コストも低いという利点を有する一方, 文脈や語順, 未知語への対応が困難であるという課題がある.
一方 fastText を用いた手法では, 教師あり学習により大量のテキストデータから単語および文書ベクトルを学習し, 文全体の特徴量に基づいて感情を判定する. fastText はサブワード情報を考慮するため, 日本語のように表記揺れや未知語が多い言語においても安定した性能を示すことが知られている.
実験では, 同一のデータセットを用い, 学習回数の増加に伴う F1 スコアの推移を比較した. その結果, 辞書形式では初期段階から一定の性能を示すものの, 学習による改善が見られず, F1 スコアはほぼ横ばいで推移した. これに対し fastText では, 学習回数の増加に伴い F1 スコアが段階的に向上し, 最終的に辞書形式を大きく上回る性能を示した.
この結果から, 文脈情報を考慮できる fastText が, 感情判定タスクにおいてより適した手法であることが示唆される.
3.7 LLM の概要
近年, 大規模言語モデル(LLM: Large Language Model)は, 自然言語処理分野において急速な発展を遂げている. LLM は大量のテキストデータを用いて学習され, 文脈理解, 文章生成, 要約, 対話応答など, 多様な言語タスクに対応可能である.
従来の言語モデルは, 主に単語や文の出現確率を推定することを目的としていた. これに対し LLM は, トランスフォーマーベースのアーキテクチャを採用することで, 文脈全体を考慮した高次の意味理解を可能としている. その結果, 人間に近い自然な文章生成や, 状況に応じた柔軟な応答が実現されている.
感情理解やメンタルサポートの分野においても, LLM は大きな可能性を有している. 単なる感情の分類にとどまらず, 利用者の発話意図や心理状態を総合的に解釈し, 適切な言葉がけや助言を生成できる点が特徴である.
一方で, LLM はモデルサイズが大きく, 推論コストや応答遅延といった課題も存在する. また, 出力の一貫性や制御性の観点から, リアルタイム性が求められるシステムへの単独利用には慎重な設計が必要である.
本研究では, LLM を感情判定の主手法としてではなく, 高次の対話生成および支援的応答を担う要素として位置づけている.
3.8 fastText と LLM の統合利用
本研究では, fastText と LLM を組み合わせた統合的な感情理解・支援システムを提案する. 両者を併用することで, 単一手法では実現が困難な即時性と柔軟性を両立することを目的としている.
fastText は軽量かつ高速に動作し, 入力文に対するネガティブ・ポジティブ判定を即座に行うことができる. この特性は, リアルタイムでのメンタル状態推定において大きな利点となる. 一方で, fastText は分類結果のみを出力するため, 利用者への具体的な言葉がけや対話生成には適していない.
そこで本研究では, fastText による感情判定結果をトリガーとして LLM を活用する構成を採用した. 具体的には, fastText により推定された感情状態を LLM に入力し, その結果を踏まえた対話応答やサポートメッセージを生成する.
この統合構成により, 軽量な感情推定と高度な言語生成を役割分担させることが可能となる. また, 不要な場面で LLM を起動しない設計とすることで, 計算資源の節約および応答遅延の低減も実現できる.
以上より, fastText と LLM の統合利用は, 実用性と表現力を兼ね備えたメンタルサポートエージェントの実現に有効であると考えられる.
