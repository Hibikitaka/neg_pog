
 \documentclass{jsarticle}
 \begin{document}
\fi
%%%%%%%%%%%%%%%%%%%%%%%%%%%%%%%%%%%%%%%%%%%%%%%%%%%%%%%%%%%%%%%%%%%%%%%%%%%%%%%%%%%%%%%%%%%%%%%%%%%%%%%%%%%%%%%%%%%%%%%%%%%%%%%%%%%%%%%%%%%%%%%%%%%%%%%%%%%%%%%%%%%%%%%%%%%%%%%%%%%%%%%%%%%%%%%%%%%%%%%%%%%%%%%%%%%%%%%%%%%%%%%%%%%%%%%%%%%%

%------------------------------------------------------------------------------------------------------------------  MindWaveMobile  --------------------------------------------------------------------------------------------------------%

\chapter{MindWaveMobile}

本章では本研究で使用するMindWaveMobileと呼ばれる簡易脳波計の概要,及び操作方法の説明を行う.
簡易脳波計を利用した研究には思考状態に関する研究~\cite{mindwave}や快・不快に関する研究~\cite{comfortable}などが行われており,メンタルを推定するために簡易脳波計が利用できるのではないかと考えた.
また本システムに脳波を利用することを考えた際,装着が容易で装着者の行動を制限しないものの方が理想である.
MindWaveMobileにはNeuroSky社の簡易脳波計に対応したライブラリやアプリケーションがあるため,研究や開発をすることがしやすい環境になっていることから本研究ではMindWaveMobileを使用する.


%----------------------------------------------------------------------------------------------------------------  MindWaveMobileの概要  -----------------------------------------------------------------------------------------------------%

\section{MindWaveMobileの概要}

MindWaveMobileとはNeuroSky社が開発した高性能センサーモジュール搭載の簡易的な脳波測定機である.


%---------------  図挿入  ---------------%
\begin{figure}[H]
 \begin{center}
 \includegraphics[scale=0.85]{./Image/MWM.png}
 \caption{MindWaveMobile}
 \label{MWM}
 \end{center}
\end{figure}
%---------------  図終了  --------------%


MindWaveMobileは脳波の数値データをPCに送信する.額のセンサーと耳の電極の2点間の電位差を図り,イヤーパッドに内蔵されているオンボードチップにより取得した脳波を解析し,通信用PCへ無線通信方式の1つであるBluetoothを用いて
送信が行われる.MindWaveMobileの特徴を以下に列挙する.


%-------------  アイテム挿入  -------------%
\begin{itemize}
  \item 測定箇所:前頭葉(国際10/20(Fp1))の1点センサー
  \item 耳たぶに基準点を設けている
  \item ドライセンサー型EEGモジュール
  \item センシングから解析までイヤーパッド内チップで行う
  \item ほとんどのプロセッサやDPSで動作可能
  \item PCへのデータ転送はBluetooth通信を用いる
  \item 512Hzでサンプリング
  \item 1秒ごとにFFT(高速フーリエ変換)をかけて各周波数成分を抽出
\end{itemize}
%-------------  アイテム終了  -------------%


サンプリング周波数512Hzであるため1秒間に512個の原脳波データを得る.原脳波のデータにFFTをかけて周波数成分を抽出し,データをディジタル信号化してPCにデータを送信する.
これ以外にも送信されるデータがありpoor\_sig\_lev(ノイズの強さ),e\_Senceメータ(NeuroSky社独自の指標)であるattention(集中度)とmeditation(瞑想度)もデータとして受け取ることが可能である.またNeuroSky社の簡易脳波計に対応したライブラリやアプリケーションがあり,研究や開発をすることがしやすい環境になっている.
FFTの際の各周波数成分範囲は以下の通りである.


%---------------  表挿入  ---------------%
\begin{table}[H]
 \begin{center}
 \normalsize
 \caption{各周波数成分範囲}
  \begin{tabular}{|c|c|c|}													\hline	
  タイプ&測定可能データ(Hz)&心理状態								\\	\hline
  $\delta$波&0.5~2.75&夢を見ない不快睡眠,ノンレム睡眠,無意識		\\	\hline
  $\theta$波&3.5~6.75&直感的,創造的,想起,空想,幻想,夢			\\	\hline
  low$\alpha$波&7.5~9.25&リラックス,ただし気だるくはない,平穏意識的		\\	\hline
  high$\alpha$波&10~11.75&リラックスしているが集中している,統合的		\\	\hline
  low$\beta$波&13~16.75&思考,自己および環境の認識					\\	\hline
  high$\beta$波&18~29.75&警戒,動揺								\\	\hline
  low$\gamma$波&31~39.75&記憶,高次精神活動						\\	\hline
  mid$\gamma$波&41~49.75&視覚情報処理							\\	\hline
  \end{tabular}
  \label{tb:eachfrequency}
 \end{center}
\end{table}
%---------------  表終了  --------------%


さらに,この脳波情報をもとに集中度,リラックス度といった精神状態が値として観測できる.
この脳波測定機はWindows,Mac OS,iPhone,Androidなど様々な機器に対応している.
対応OSを表\ref{tb:os}で示す.


%---------------  表挿入  ---------------%
\begin{table}[H]
 \begin{center}
 \normalsize
 \caption{MindWaveMobile対応OS一覧}
  \begin{tabular}{|c|c|c|}																	\hline
  &対応機種&Windows XP,Windows Vista,Windows7,8,8.1								\\	\cline{2-3}
  対応OS(PC)&CPU&Intel Core 2 Duo												\\	\cline{2-3}
  &メモリ&1GB RAM																\\	\hline
  &対応機種&Mac OS X10.5.8~10.9													\\	\cline{2-3}
  対応OS(Mac)&CPU&Intel processor 内蔵											\\	\cline{2-3}
  &メモリ&1GB RAM																\\	\hline
  対応OS(iOS)&対応機種&iPod touch(2世代~4世代),iPhone3GS,iPhone,iPad2,iPad他		\\	\hline
  \end{tabular}
  \label{tb:os}
 \end{center}
\end{table}
%---------------  表終了  --------------%


%---------------------------------------------------------------------------------------------------------------  ThinkGear ASICモジュール  ----------------------------------------------------------------------------------------------------%

\section{ThinkGear ASICモジュール}

ThinkGear ASICモジュール(TGAM)とは,MindWaveMobileに搭載されている脳波センサーモジュールである.
TGAMの機能として取得した微弱な電気信号からノイズ除去を行い,信号を増幅しデジタル信号への変化を行うことで観測が可能となる.
また自社開発のアルゴリズムeSenceを用いることで集中度,リラックス度のリアルタイム表示行うなどの機能を搭載している.


%---------------  図挿入  ---------------%
\begin{figure}[H]
 \begin{center}
 \includegraphics[scale=0.85]{./Image/asic.jpeg}
 \caption{Think Gear ASICモジュール}
 \label{asic}
 \end{center}
\end{figure}
%---------------  図終了  --------------%


%----------------------------------------------------------------------------------------------------------------------  eSence  ------------------------------------------------------------------------------------------------------------%

\subsubsection{eSence}

eSenceとはTGAMに搭載されている心理状態を描写するアルゴリズムである.脳波信号を増幅し環境雑音と筋肉の動きを除去したのち,eSenceアルゴリズムを残った信号に適応することで,集中度・リラックス度メーター値に変換される.集中度・リラックス度メーター値は正確な数値を示すものでなく,
活動範囲を示すあいまいなものであり,それぞれ0~100の範囲で動的に観測が行える.この数値が高いほど評価が高く設定されている.


%---------------  表挿入  ---------------%
\begin{table}[H]
 \begin{center}
 \normalsize
 \caption{メータの範囲設定}
  \begin{tabular}{|c|c|}		\hline
  数値範囲&状態		\\	\hline
  100~80&非常に高い	\\	\hline
  80~60&高い		\\	\hline
  60~40&普通		\\	\hline
  40~20&低い		\\	\hline
  20~0&非常に低い	\\	\hline
  \end{tabular}
  \label{tb:meter}
 \end{center}
\end{table}
%---------------  表終了  --------------%


集中度は高度に集中している時や,安定した精神活動時に数値が上昇し,注意散漫,集中力の欠如,不安などの状態だと数値が下降する.リラックス度はリラックスすることで値が上昇し,不安,動揺,感覚の刺激などで値が低下する.さらに目を閉じることで値の上昇が見込まれる.


%----------------------------------------------------------------------------------------------------------------  MindWaveMobileの機能  -----------------------------------------------------------------------------------------------------%
\newpage
\section{MindWaveMobileの機能}

MindWaveMobile本体の部分毎の名称と,機能について図\ref{MWMfig}で説明を行う.


%---------------  図挿入  ---------------%
\begin{figure}[H]
 \begin{center}
 \includegraphics[scale=0.7]{./Image/MWM_fig.png}
 \caption{MindWaveMobile本体}
 \label{MWMfig}
 \end{center}
\end{figure}
%---------------  図終了  --------------%


%----------------------------------------------------------------------------------------------------------------------  スイッチ  ------------------------------------------------------------------------------------------------------------%

\subsubsection{スイッチ}

スイッチのon,offでMindWaveMobileの起動,終了を行う.


%-----------------------------------------------------------------------------------------------------------  導電部,イヤークリップ,センサーアーム  -----------------------------------------------------------------------------------------------%

\subsubsection{導電部,イヤークリップ,センサーアーム}

各部位の名称.脳波測定を行う際,装着,動作確認を行う必要がある.詳しくは次項で説明する.


%---------------------------------------------------------------------------------------------------------------------  LEDライト  -----------------------------------------------------------------------------------------------------------%

\subsubsection{LEDライト}

LEDライトの点滅状態によりMindWaveMobileの状態を知ることができる.
内訳を表\ref{tb:led}で表す.


%---------------  表挿入  ---------------%
\begin{table}[H]
 \begin{center}
 \normalsize
 \caption{LEDの点滅による内訳}
  \begin{tabular}{|c|c|c|}												\hline	
  LEDライト&MindWaveMobileの状態&意味							\\	\hline
  off&電源off&電源が入っていない									\\	\hline
  青色2連続点滅&ペアリングモード&デバイスとのペアリング準備が完了した	\\	\hline
  赤&ペアリング失敗&デバイスをペアリングする必要がある				\\	\hline
  青点滅&検索&MindWaveMobileが接続するデバイスを検索している		\\	\hline
  青&接続&MindWaveMobileが接続している							\\	\hline
  赤色2連続点滅&電池消耗&電池の交換が必要					\\	\hline
  \end{tabular}
  \label{tb:led} 
 \end{center}
\end{table}
%---------------  表終了  --------------%


%----------------------------------------------------------------------------------------------------------------------  ペアリング  -----------------------------------------------------------------------------------------------------------%

\subsubsection{ペアリング}

Bluetooth接続を行う際,機器同士を登録しておく必要がある.その登録の名称をペアリングと呼ぶ.


%--------------------------------------------------------------------------------------------------------------  MindWaveMobileの装着方法  ---------------------------------------------------------------------------------------------------%

\section{MindWaveMobileの装着方法}

MindWaveMobileの装着方法について下記に記す.


%-------------  アイテム挿入  -------------%
\begin{enumerate}
  \item スイッチを入れる.(接続機器と接続ができていない場合,次項のMindWaveMobileの接続方法を参照する)
  \item 装着者の頭部の大きさに合わせて,ヘッドバンドとセンサーアームを調節する.
  \item 導電部が頭皮に密着するように装着を行う.(ヘッドバンドを調節することで,測定装置を固定する)
  \item イヤークリップを耳朶に装着する.
\end{enumerate}
%-------------  アイテム終了  -------------%


%--------------------------------------------------------------------------------------------------------------  MindWaveMobileの接続方法  ---------------------------------------------------------------------------------------------------%

\newpage
\section{MindWaveMobileの接続方法}

MindWaveMobileはAndroid,iPhoneなどの携帯端末や,Windows,MacOSなどのPC間でBluetooth接続を行うことでのうは情報を読み取ることができる.本稿では,本研究の開発環境であるWindows10の接続方法について記載する.


%-----------------------------------------------------------------------------------------------------------------------  step1  ------------------------------------------------------------------------------------------------------------%

\subsubsection{step1}

ヘッドセットの電源スイッチをONにし,LEDが青色2連続するとペアリングモードが自動的にスタートする.スタートしない場合,電源スイッチをONの方向に3秒スライドさせる.


%-----------------------------------------------------------------------------------------------------------------------  step2  ------------------------------------------------------------------------------------------------------------%

\subsubsection{step2}

以下の手順に従いペアリングを行う.


%-------------  アイテム挿入  -------------%
\begin{enumerate}
  \item スタートをクリックしコントロールパネルへ
  \item 右側上端部分の表示方法から”小さなアイコン”をクリック
  \item デバイスとプリンターを開き,デバイスを追加する.
  \item MindWaveMobileを選択し,次へをクリックする.
  \item ペアリングが完了したら閉じるをクリックする.
\end{enumerate}
%-------------  アイテム終了  -------------%


%-----------------------------------------------------------------------------------------------------------------------  step3  ------------------------------------------------------------------------------------------------------------%

\subsubsection{step3}

接続完了後,Bluetoothソフトウェアに”ペアリング済み”,または”接続完了”と表示され,ヘッドセットのLEDが青色点滅する.
一度ヘッドセットとのペアリングが完了すると,今度はヘッドセットのスイッチをONにすることで,接続が可能になる.


%%%%%%%%%%%%%%%%%%%%%%%%%%%%%%%%%%%%%%%%%%%%%%%%%%%%%%%%%%%%%%%%%%%%%%%%%%%%%%%%%%%%%%%%%%%%%%%%%%%%%%%%%%%%%%%%%%%%%%%%%%%%%%%%%%%%%%%%%%%%%%%%%%%%%%%%%%%%%%%%%%%%%%%%%%%%%%%%%%%%%%%%%%%%%%%%%%%%%%%%%%%%%%%%%%%%%%%%%%%%%%%%%%%%%%%%%%%%
\expandafter\ifx\csname ifdraft\endcsname\relax
  \end{document}
\fi