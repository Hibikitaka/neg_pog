\documentclass{jsarticle}
 \begin{document}
\fi
%%%%%%%%%%%%%%%%%%%%%%%%%%%%%%%%%%%%%%%%%%%%%%%%%%%%%%%%%%%%%%%%%%%%%%%%%%%%%%%%%%%%%%%%%%%%%%%%%%%%%%%%%%%%%%%%%%%%%%%%%%%%%%%%%%%%%%%%%%%%%%%%%%%%%%%%%%%%%%%%%%%%%%%%%%%%%%%%%%%%%%%%%%%%%%%%%%%%%%%%%%%%%%%%%%%%%%%%%%%%%%%%%%%%%%%%%%%%

%------------------------------------------------------------------------------------------------------------------  脳波の概要 --------------------------------------------------------------------------------------------------------%

\chapter{第2章 脳波の概要}

2.1 脳波とは
脳波とは,脳内の神経細胞,特に大脳皮質に存在するニューロン群の電気的活動を,頭皮上から計測した電位変動の総和である.個々のニューロンが発生させる電気信号は微弱であるが,多数のニューロンが同期的に活動することで,頭皮上でも計測可能な電位変動として観測される.これを脳波(Electroencephalogram,EEG)と呼ぶ.
脳波は時間とともに連続的に変化する信号であり,振幅および周波数の違いによって特徴付けられる.一般に,脳波は周波数帯ごとに分類され,それぞれが人間の覚醒状態や認知活動,感情状態と深く関係していることが知られている.このため,脳波は医学分野のみならず,心理学,認知科学,ヒューマンインタフェース,感情認識などの分野においても重要な指標として用いられている.
本研究では,脳波をメンタル状態推定のための主要なバイタルデータの一つとして位置付けており,特に脳波の周波数特性と心理状態との対応関係に着目する.
2.2 脳波の周波数帯分類
一般に,脳波はその周波数帯域に基づいて,δ波,θ波,α波,β波などに分類される.これらの分類は,脳の活動状態を理解するための基本的な枠組みであり,多くの研究で共通して用いられている.
2.2.1 δ波
δ波は,およそ0.5〜4Hzの低周波数帯に属する脳波であり,主に深い睡眠状態において顕著に観測される.特に,ノンレム睡眠の深い段階では,δ波が優勢となることが知られている.
覚醒時においてδ波が増加する場合,極度の疲労,意識レベルの低下,または集中力の著しい低下が生じている可能性が示唆される.そのため,δ波は人間の意識レベルや覚醒度を評価する指標として用いられることがある.
メンタル状態との関係においては,δ波は直接的に感情を反映するというよりも,心身の疲労度や回復状態を示す指標として位置付けられることが多い.
2.2.2 θ波
θ波は,およそ4〜8Hzの周波数帯に属する脳波であり,浅い睡眠状態や,覚醒と睡眠の中間状態において多く観測される.また,リラックス状態や瞑想状態,あるいは内省的な思考を行っている際にもθ波が増加することが報告されている.
一方で,覚醒時においてθ波が過剰に出現する場合,注意力の低下や集中力不足,精神的疲労の兆候である可能性がある.このため,θ波は集中状態と非集中状態を識別する上で重要な指標とされている.
メンタル状態との関係においては,θ波はリラックスと集中低下の両側面を持つ波形であり,他の周波数帯との相対的な関係を考慮することが重要である.
2.2.3 α波
α波は,およそ8〜13Hzの周波数帯に属する脳波であり,安静覚醒状態において最も典型的に観測される.目を閉じてリラックスしている状態ではα波が優勢となり,逆に目を開けたり,外部刺激に注意を向けたりするとα波は減衰することが知られている.
α波は,リラックス状態や精神的安定と深く関係しており,ストレスが低く,落ち着いた状態で増加する傾向がある.一方で,過度にα波が優勢な状態は,外界への注意が低下している可能性を示す場合もある.
本研究においては,α波をメンタルの安定度やリラックス度を評価する重要な指標として扱う.
2.2.4 β波
β波は,およそ13〜30Hzの周波数帯に属する脳波であり,覚醒時の思考活動や問題解決,注意集中時に増加する.学習作業や会話,意思決定を行っている際には,β波が優勢となることが多い.
一方で,β波が過剰に出現する状態は,精神的緊張や不安,ストレスの増加を示唆することがある.特に,高周波成分のβ波は,過覚醒状態や焦燥感と関連付けられることが多い.
このように,β波は集中と緊張の両側面を持つ脳波であり,α波やθ波との比率関係がメンタル状態の推定において重要となる.
%---------------  表挿入  ---------------%
\begin{table}[H]
    \begin{center}
    \normalsize
    \caption{各周波数成分範囲}
     \begin{tabular}{|c|c|c|}													\hline	
     タイプ&測定可能データ(Hz)&心理状態								\\	\hline
     $\delta$波&0.5~2.75&夢を見ない不快睡眠,ノンレム睡眠,無意識		\\	\hline
     $\theta$波&3.5~6.75&直感的,創造的,想起,空想,幻想,夢			\\	\hline
     low$\alpha$波&7.5~9.25&リラックス,ただし気だるくはない,平穏意識的		\\	\hline
     high$\alpha$波&10~11.75&リラックスしているが集中している,統合的		\\	\hline
     low$\beta$波&13~16.75&思考,自己および環境の認識					\\	\hline
     high$\beta$波&18~29.75&警戒,動揺								\\	\hline
     low$\gamma$波&31~39.75&記憶,高次精神活動						\\	\hline
     mid$\gamma$波&41~49.75&視覚情報処理							\\	\hline
     \end{tabular}
     \label{tb:eachfrequency}
    \end{center}
   \end{table}
   %---------------  表終了  --------------%
2.3 脳波とメンタル状態の関係
脳波とメンタル状態との関係は,単一の周波数帯のみで評価されるものではなく,複数の周波数成分の相対的なバランスによって特徴付けられる.例えば,α波が優勢でβ波が抑制されている状態は,リラックスかつ安定した心理状態を示すと考えられる.一方で,β波が優勢でα波が低下している状態は,強い集中や緊張状態を示す可能性がある.
また,θ波とβ波の比率は,注意力や集中度を評価する指標として用いられることがあり,教育分野や認知研究においても活用されている.これらの比率指標は,個人差が存在するため,絶対値ではなく,相対的変化として扱うことが重要である.
本研究では,脳波の各周波数帯のパワー値およびそれらの比率を特徴量として抽出し,機械学習モデルに入力することで,メンタル状態の推定を行う.
2.4 正常脳波の定義と年齢による特徴
正常脳波とは,器質的脳疾患や明らかな神経学的異常が認められない状態において観測される脳波活動を指す.正常脳波は,被験者の年齢,覚醒レベル,精神状態などによって変化するため,一律の波形をもって定義されるものではなく,年齢層ごとの特徴を考慮する必要がある.
2.4.1 小児における正常脳波の特徴
小児期における脳波は,成人と比較して低周波成分が優勢であることが特徴である.特に,乳幼児ではδ波やθ波が多く観測され,脳の成熟に伴って徐々に高周波成分が増加していく.
学童期に入ると,θ波の割合が減少し,α波が後頭部を中心に出現し始める.しかし,この段階においても成人と比較するとα波の周波数はやや低く,振幅が大きい傾向がある.このような特徴は,脳機能が発達途上にあることを反映したものであり,必ずしも異常を示すものではない.
2.4.2 成人における正常脳波の特徴
成人における正常脳波では,安静覚醒状態において後頭部優位のα波が明瞭に観測されることが一般的である.α波の周波数は約8〜13Hzの範囲に収まり,左右差が小さいことが正常所見とされる.
覚醒状態での作業や思考時にはβ波が増加し,集中状態と対応する.成人脳波では,これらの周波数帯が状況に応じて柔軟に変化することが特徴であり,脳機能の成熟と安定性を反映している.
2.4.3 高齢者における正常脳波の特徴
高齢者においては,加齢に伴いα波の周波数が低下し,θ波の出現頻度が増加する傾向がある.また,全体的な脳波振幅の低下や,左右差の増大が観測される場合もある.
これらの変化は,加齢に伴う生理的変化として一定範囲内で認められるものであり,必ずしも病的所見を意味するものではない.しかし,θ波やδ波が過剰に出現する場合には,認知機能低下や脳機能障害の可能性を考慮する必要がある.
2.5 脳波計測におけるアーチファクトの詳細
脳波計測では,脳神経活動以外に由来する電気信号が混入することがあり,これらは総称してアーチファクトと呼ばれる.アーチファクトは脳波信号の解釈を困難にし,メンタル状態推定や機械学習モデルの性能低下を招くため,その発生要因と特性を正しく理解することが重要である.
2.5.1 脈波
脈波によるアーチファクトは,血管の拍動に伴う微小な電位変動や,頭皮表面の物理的変形によって生じる.特に,電極付近を走行する血管の拍動は,低周波成分として脳波に混入することがある.
このアーチファクトは,心拍数と同期した周期的な波形として現れることが多く,δ波やθ波帯域と重なるため,覚醒度や疲労状態の評価に影響を及ぼす可能性がある.対策としては,電極の装着位置を工夫することや,心拍データと同期させた信号処理による除去が有効とされている.
2.5.2 眼球運動
眼球運動によるアーチファクトは,眼球が正負の電位を持つ電気双極子として振る舞うことに起因する.眼球が上下左右に動くことで,前頭部を中心に大きな電位変動が発生し,脳波に混入する.
このアーチファクトは,主に低周波成分として現れ,特に前頭部電極に強く影響する.集中状態やリラックス状態の評価において,θ波やα波の誤検出を引き起こす要因となる.計測時には視線移動を最小限に抑えるよう指示し,解析段階ではEOG成分の除去手法を用いることが一般的である.
2.5.3 発汗
発汗によるアーチファクトは,皮膚表面の水分量変化に伴い,電極と皮膚間のインピーダンスが変動することで生じる.特に,緊張やストレス状態において発汗が増加すると,脳波信号のベースラインが不安定になる.
この影響は,緩やかな電位ドリフトとして観測されることが多く,低周波帯域の解析に影響を与える.対策としては,電極装着前の皮膚清拭や,長時間計測時における信号の正規化処理が挙げられる.
2.5.4 まばたき
まばたきによるアーチファクトは,眼瞼運動に伴う筋電位と,眼球電位変動の複合的な影響によって生じる.まばたき時には瞬間的に大振幅の電位変動が発生し,脳波信号中に鋭いピークとして現れる.
このアーチファクトは,前頭部電極に顕著であり,短時間であっても解析結果に大きな影響を与える可能性がある.解析においては,ピーク検出や独立成分分析などを用いて除去されることが多い.
2.5.5 不随意運動
不随意運動によるアーチファクトは,筋肉の無意識的な収縮によって生じる筋電位が原因である.特に,顔面筋や首周辺の筋肉は脳波電極に近接しているため,その影響が顕著である.
筋電位は高周波成分を多く含み,β波帯域やそれ以上の周波数帯と重なるため,集中状態や緊張状態の誤判定を引き起こす要因となる.被験者に対してリラックスを促すことや,高周波成分を抑制するフィルタ処理が対策として用いられる.
2.5.6 体動
体動によるアーチファクトは,被験者の姿勢変化や頭部の動きにより,電極位置がずれたり,電極と皮膚の接触状態が変化することで発生する.
この種のアーチファクトは,突発的かつ大振幅のノイズとして観測されることが多く,解析においてはデータ区間の除外が必要となる場合もある.計測時には,安定した姿勢を保つよう被験者に指示し,装置の固定を十分に行うことが重要である.
2.5.7 静電誘導
静電誘導によるアーチファクトは,周囲の帯電物体や人体表面の静電気が,電極やケーブルに誘導されることで発生する.特に,乾燥した環境では静電気の影響が顕著になる.
このアーチファクトは,突発的なノイズや基線の揺らぎとして観測されることが多い.対策としては,計測環境の湿度管理や,導電性素材を用いたシールドが有効である.
2.5.8 電磁誘導
電磁誘導によるアーチファクトは,周囲の電子機器や電源ケーブルから発生する電磁場が,脳波計測システムに干渉することで生じる.
代表的な例として,商用電源に起因する周期的なノイズが挙げられる.これらは特定周波数に集中して現れるため,ノッチフィルタなどの周波数選択的処理が有効である.
2.5.9 分極電圧
分極電圧は,電極と皮膚の界面において電気化学反応が生じることで発生する電位差である.この電圧は時間とともに変動し,脳波信号の基線ドリフトを引き起こす.
分極電圧の影響を低減するためには,非分極電極の使用や,適切な電極ジェルの塗布が重要である.
2.5.10 静電気
静電気によるアーチファクトは,被験者や周囲環境に蓄積された電荷が放電することで発生する.この影響は,突発的な大振幅ノイズとして現れることが多い.
静電気対策としては,接地の徹底や,合成繊維の使用を避けることが有効である.
2.5.11 光電効果
光電効果によるアーチファクトは,強い光が電極やセンサに照射されることで,光電変換に起因する電流が発生する現象である.特に,光センサと併用する装置では注意が必要である.
このアーチファクトは,照明条件の変化と同期して観測されることがあり,計測環境の光制御が重要となる.
2.7 BCIおよびBMIの概要
BCI(Brain Computer Interface)およびBMI(Brain Machine Interface)は,脳活動を直接利用して外部機器やシステムと情報をやり取りする技術である.これらの技術は,医療,福祉,ヒューマンインタフェース分野において注目されている.
2.7.1 P300スペラー
P300スペラーは,事象関連電位の一種であるP300成分を利用したBCIシステムである.被験者が特定の刺激に注意を向けた際に出現するP300成分を検出し,文字入力などを行うことが可能である.
この方式は,重度運動障害者の意思伝達手段として実用化が進んでおり,比較的少ない学習コストで利用可能である点が特徴である.
2.7.2 運動出力型BMI
運動出力型BMIは,運動想起に伴う脳波変化を利用して,ロボットアームやカーソルなどを制御する方式である.被験者が手や足の動きを想像することで,対応する脳波パターンが出現し,これを機械学習により識別する.
この方式は,リハビリテーションや義手制御などへの応用が期待されている.
2.7.3 感覚入力型BMIおよび直接操作型BMI
感覚入力型BMIは,外部刺激を脳へフィードバックすることで感覚を補完または拡張する技術である.一方,直接操作型BMIは,脳活動を用いて外部機器を直感的に操作することを目的とする.
これらの技術は,人間と機械の関係性を大きく変える可能性を持ち,本研究におけるAIエージェントとの相互作用設計にも示唆を与えるものである.
