\expandafter\ifx\csname ifdraft\endcsname\relax
 \documentclass{jsarticle}
 \begin{document}
\fi
%%%%%%%%%%%%%%%%%%%%%%%%%%%%%%%%%%%%%%%%%%%%%%%%%%%%%%%%%%%%%%%%%%%%%%%%%%%%%%%%%%%%%%%%%%%%%%%%%%%%%%%%%%%%%%%%%%%%%%%%%%%%%%%%%%%%%%%%%%%%%%%%%%%%%%%%%%%%%%%%%%%%%%%%%%%%%%%%%%%%%%%%%%%%%%%%%%%%%%%%%%%%%%%%%%%%%%%%%%%%%%%%%%%%%%%%%%%%

\chapter{脳波計測装置 Muse S}
本章では本研究で使用するMuse Sと呼ばれる脳波計測装置の概要.及び操作方法の説明を行う.

%---------------------------------------------------------------------------------------------------------------- Muse S の概要  --------------------------------------------------------------------------------------------------------------%
\section{Muse S の概要}
本研究では, 被験者のメンタル状態を推定するための生体情報として脳波を用いる. 
従来, 脳波計測には医療用 EEG 装置が用いられることが多く, 高精度な計測が可能である一方で, 装着の煩雑さや高コスト, 実験環境の制約といった課題が存在していた. これらの課題を踏まえ, 一般消費者向けの簡易型脳波計測装置が開発されている.
本研究では, 比較的高い計測精度と装着性を兼ね備えたヘッドバンド型脳波計測装置である Muse S を使用する.

%---------------------------------------------------------------------------------------------------------------- Muse S の基本構成  --------------------------------------------------------------------------------------------------------------%
\section{Muse S の基本構成}
Muse S は, カナダの InteraXon 社によって開発されたヘッドバンド型の脳波計測装置である. 本装置は, 前頭部および左右側頭部に配置された複数の電極を用いて脳波を計測する構成となっている.
Muse S の外観を図4.1に示す.

%---------------  図挿入  ---------------%
\begin{figure}[H]
 \begin{center}
 \includegraphics[scale=0.85]{./Image/MuseS.png}
 \caption{Muse S}
 \label{MuseS}
 \end{center}
\end{figure}
%---------------  図終了  --------------%
図4.1 Muse S の外観
本装置は軽量であり, 長時間装着しても被験者の負担が小さい設計となっている. また, ドライ電極を採用しているため, 電極ペーストやジェルを使用する必要がなく, 簡便に脳波計測を行うことが可能である.

%---------------------------------------------------------------------------------------------------------------- Muse S の主な機能  --------------------------------------------------------------------------------------------------------------%
\section{Muse S の主な機能}
Muse S は, 脳波計測を中心とした複数の機能を備えたウェアラブルデバイスである.
第一に, 脳波のリアルタイム計測機能が挙げられる. Muse S は, $\delta$波, $\theta$ 波, $\alpha$ 波, $\beta$ 波といった周波数帯域別の脳波成分を抽出し, 一定周期で外部デバイスへ送信することが可能である. これにより, 被験者の覚醒状態や集中状態, リラックス状態の変化を即時に把握できる.
第二に, 生脳波データの取得機能である. 周波数帯域別データに加え, 電極ごとの生 EEG 波形を取得できるため, 後処理による特徴量抽出や解析にも対応可能である.
第三に, 無線通信機能である. Muse S は Bluetooth Low Energy(BLE) を用いた通信方式を採用しており, PC やスマートフォンとワイヤレスで接続できる.

%----------------------------------------------------------------------------------------------------------------  計測可能な脳波データ  --------------------------------------------------------------------------------------------------------------%
\section{計測可能な脳波データ}
Muse S では, 主に以下の周波数帯域に分類された脳波データを取得することが可能である.$\delta$ 波,$\theta$ 波,$\alpha$ 波,$\beta$ 波
これらの周波数帯域は, 被験者の精神状態や覚醒度と密接な関係を有している. $\alpha$ 波は安静時やリラックス状態で増加する傾向があり, $\beta$ 波は集中や緊張状態において顕著となる. $\theta$ 波は瞑想状態や浅い睡眠時に, $\delta$ 波は深い睡眠時に主に観測される.
本研究では, これらの周波数帯域ごとのパワー値および比率を特徴量として利用する.

\section{Muse S の装着方法}
Muse S はヘッドバンド型の構造を採用しており, 比較的容易に装着することが可能である. 正確な脳波計測を行うためには, 電極が適切な位置で皮膚と接触していることが重要となる.
装着時には, 前頭部の電極が左右の眉毛上付近に位置するように調整し, 側頭部の電極が耳の上付近に密着するように装着する. Muse S の装着例を図4.2に示す.

図4.2 Muse S の装着例

ドライ電極を使用しているため, 装着準備に要する時間は短いが, 電極の接触状態が不十分な場合, ノイズの混入や信号品質の低下が生じる可能性がある. そのため, 計測前には信号品質を確認することが重要である.

%---------------------------------------------------------------------------------------------------------------- Bluetooth 接続方法と BLED112 ドングル  --------------------------------------------------------------------------------------------------------------%
\section{Bluetooth 接続方法と BLED112 ドングル}
Muse S は Bluetooth Low Energy(BLE) による無線通信を用いて外部デバイスと接続する. PC 上で安定したデータ取得を行うため, 本研究では BLE 通信専用の USB ドングルである BLED112 を使用した.
Muse S と PC の接続構成を図4.3に示す.

図4.3 Muse S と PC の Bluetooth 接続構成

接続手順としては, まず PC の USB ポートに BLED112 ドングルを接続する. 次に Muse S の電源を入れ, ペアリング可能な状態に設定する. 専用のストリーミングツールを起動し, 接続可能な BLE デバイスの検索を行った後, Muse S を選択して接続を行う.
接続が成功すると, Muse S から送信される脳波データがリアルタイムで PC に受信される.

%---------------------------------------------------------------------------------------------------------------- データ取得および保存方法  --------------------------------------------------------------------------------------------------------------%
\section{データ取得および保存方法}
取得した脳波データは, 時系列データとして PC 上に保存される. 本研究では, 一定時間ごとにデータを区切り, 周波数帯域別の平均値や比率を算出することで, メンタル状態推定に用いる特徴量を生成した.
この構成により, 被験者の状態変化をリアルタイムに追跡しつつ, 後処理による解析も可能となる.

%---------------------------------------------------------------------------------------------------------------- 本研究における Muse S の位置づけ  --------------------------------------------------------------------------------------------------------------%
\section{本研究における Muse S の位置づけ}
本研究において Muse S は, 被験者の内的状態を客観的に取得するための主要なセンサとして位置づけられる. 自己申告やテキスト入力のみに依存せず, 脳波という生体信号を用いることで, より潜在的なメンタル状態を推定できる点が本研究の特徴である.
また, 装着負担が小さい Muse S は, 長時間計測や日常環境での利用にも適しており, 将来的な実用システムへの応用可能性も高い.
